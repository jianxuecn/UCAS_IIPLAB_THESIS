%---------------------------------------------------------------------------%
%->> Frontmatter
%---------------------------------------------------------------------------%
%-
%-> 生成封面
%-
\maketitle% 生成中文封面
\MAKETITLE% 生成英文封面
%-
%-> 作者声明
%-
\makedeclaration% 生成声明页
%-
%-> 中文摘要
%-
\intobmk\chapter*{摘\quad 要}% 显示在书签但不显示在目录
\setcounter{page}{1}% 开始页码
\pagenumbering{Roman}% 页码符号

本文是中国科学院大学学位论文模板ucasthesis的使用说明文档。主要内容为介绍\LaTeX{}文档类ucasthesis的用法,以及如何使用\LaTeX{}快速高效地撰写学位论文。

中文摘要、英文摘要、目录、论文正文、参考文献、附录、致谢、攻读学位期间发表的学术论文与其他相关学术成果等均须由另页右页(奇数页)开始。

\keywords{中国科学院大学,学位论文,\LaTeX{}模板}% 中文关键词
%-
%-> 英文摘要
%-
\intobmk\chapter*{Abstract}% 显示在书签但不显示在目录

This paper is a help documentation for the \LaTeX{} class ucasthesis, which is  a thesis template for the University of Chinese Academy of Sciences. The main content is about how to use the ucasthesis, as well as how to write thesis efficiently by using \LaTeX{}.

Chinese abstracts, English abstracts, table of contents, the main contents, references, appendix, acknowledgments, author's resume and academic papers published during the degree study and other relevant academic achievements must start with another right page (odd-numbered page).

\KEYWORDS{University of Chinese Academy of Sciences (UCAS), Thesis, \LaTeX{} Template}% 英文关键词

\pagestyle{enfrontmatterstyle}%
\cleardoublepage\pagestyle{frontmatterstyle}%

%---------------------------------------------------------------------------%
